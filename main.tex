\documentclass[12pt,a4paper]{report}

% -------------------------
% Packages
% -------------------------
\usepackage[a4paper,margin=1in]{geometry}
\usepackage{graphicx}
\usepackage{booktabs}
\usepackage{amsmath,amssymb}
\usepackage{siunitx}
\usepackage{hyperref}
\usepackage{caption}
\usepackage{subcaption}
\usepackage{float}
\usepackage{enumitem}
\usepackage{natbib}

\hypersetup{
  colorlinks=true,
  linkcolor=blue,
  citecolor=blue,
  urlcolor=blue
}

\sisetup{
  round-mode=places,
  round-precision=2
}

% -------------------------
% Helpful macros
% -------------------------
\newcommand{\state}{s}
\newcommand{\action}{a}

\begin{document}

\begin{titlepage}
  \centering

  \includegraphics[width=0.28\textwidth]{figures/aastu.png}\par

  {\Large Addis Ababa Science and Technology University\par}
  \vspace{0.5cm}
  {\large Department of Electrical and Computer Engineering\par}
  \vspace{1.5cm}

  {\huge\bfseries Stochastic Traffic Modeling for Cloud Resource Allocation\par}
  \vspace{0.3cm}
  {\Large Trace-Driven Simulation with Poisson Baselines and an MDP Autoscaler\par}
  \vspace{1.5cm}

  {\large \textbf{Course:} Stochastic and Random Processes\par}
  \vspace{0.2cm}
  {\large \textbf{Instructor:} Dr.\ Sultan\par}
  \vspace{1.5cm}

  {\large \textbf{Prepared by:}\par}
  \vspace{0.2cm}
  {\large Kaleab Tadesse\par}
  {\large ID: FTP0848/14\par}

  \vfill

  {\large \today\par}
\end{titlepage}


\pagenumbering{roman}

\begin{abstract}
Cloud resource allocation must balance cost and performance under uncertain and bursty workloads.
This project models cloud traffic using stochastic methods and optimizes autoscaling decisions
using a Markov Decision Process (MDP) formulation. We build a trace-driven workload from the
Google Cluster Trace (2011) by extracting task arrivals, CPU/memory requests, and runtimes from
raw event logs. We implement an event-driven simulator in Python to evaluate (i) static provisioning,
(ii) a queue/backlog threshold autoscaler, and (iii) an MDP-based autoscaler trained on an aggregated
backlog model. On the busiest 2-hour trace window (112,434 tasks), the MDP policy achieves markedly
lower waiting times and SLA violation rates compared to a fixed-capacity baseline, at a comparable
VM-hour cost to heuristic autoscaling. We additionally apply the same trace-driven simulation pipeline to an Alibaba OpenB pod trace (Kaggle),
selecting the busiest 24-hour window (681 pods) due to the dataset’s lower event density. This provides a
second-dataset validation of the cost--performance trade-offs observed on Google.
\end{abstract}

\tableofcontents
\listoffigures
\listoftables

\clearpage
\pagenumbering{arabic}

% ============================================================
\chapter{Introduction}
Cloud computing platforms provision CPU and memory resources on demand. Workloads are highly
dynamic, so providers must choose resource allocations that satisfy performance goals (e.g.,
low queueing delay) while minimizing operational cost (e.g., VM-hours). This motivates (1)
stochastic modeling of uncertain arrivals and demands, and (2) principled control methods such
as Markov Decision Processes (MDPs).

\section{Project goals and scope}
This project focuses on:
\begin{itemize}[leftmargin=1.2cm]
  \item building a \textbf{trace-driven simulation} using the Google Cluster Trace dataset;
  \item using \textbf{Poisson processes} as a baseline stochastic traffic model;
  \item formulating and evaluating an \textbf{MDP autoscaling} policy;
  \item comparing policies under a cost--performance trade-off (VM-hours vs delay/SLA).
\end{itemize}

\section{Key simplifications (time-constrained project)}
To deliver an end-to-end system within a tight project schedule and laptop compute constraints,
we apply the following simplifications while preserving the core trade-off:
\begin{enumerate}[leftmargin=1.2cm]
  \item \textbf{Homogeneous VM model:} all VMs have normalized capacities
  \(C^{cpu}=1\), \(C^{mem}=1\).
  \item \textbf{Requests-based admission:} tasks consume requested CPU/memory
  rather than actual usage (consistent with many scheduling approaches).
  \item \textbf{Clean lifecycle filtering:} we keep tasks with a single unambiguous
  \texttt{SUBMIT}\(\rightarrow\)\texttt{SCHEDULE}\(\rightarrow\)\texttt{FINISH} lifecycle and ignore
  evictions/retries/preemptions.
  \item \textbf{Simple placement:} FIFO queue with a first-fit placement across VMs.
  \item \textbf{Control discretization:} autoscaler decisions occur every \(\Delta=60\) seconds.
\end{enumerate}

% ============================================================
\chapter{Background}
\section{Stochastic workload modeling and Poisson processes}
A common baseline for random arrivals is the Poisson process with rate \(\lambda\), implying
exponentially distributed inter-arrival times and independent increments. Poisson models are
mathematically tractable and often used as a first-order approximation in queueing analysis.
However, production traces frequently exhibit burstiness and correlations, motivating trace-driven
simulation and/or non-homogeneous extensions.

\section{Markov Decision Processes (MDP)}
An MDP is defined by \(\langle S, A, P, R, \gamma \rangle\), where \(S\) is the set of states, \(A\)
the set of actions, \(P\) transition probabilities, \(R\) rewards, and \(\gamma\in(0,1)\) a discount
factor. In autoscaling, the state summarizes current load and capacity; actions add/remove VMs; and
the reward encodes the cost--performance objective.

% ============================================================
\chapter{Dataset: Google Cluster Trace (2011)}
\section{Source}
We use the Google Cluster Trace (2011, version 2), which records task lifecycle events and
resource requests across a large production cluster. The dataset is provided as sharded
\texttt{task\_events} CSV files in Google Cloud Storage \citep{googleclustertrace}.

\section{Raw schema used}
From \texttt{task\_events}, we use:
\begin{itemize}[leftmargin=1.2cm]
  \item time (microseconds since trace start),
  \item job ID, task index,
  \item event type (submit/schedule/finish),
  \item CPU request and memory request (normalized fractions).
\end{itemize}

\section{Preprocessing pipeline}
We convert raw event logs into a compact task table:
\[
(\text{arrival\_time},\ \text{runtime},\ cpu\_req,\ mem\_req).
\]
Definitions:
\begin{itemize}[leftmargin=1.2cm]
  \item arrival time \(t_a\): time of \texttt{SUBMIT} event,
  \item start time \(t_s\): time of first \texttt{SCHEDULE} after submit,
  \item finish time \(t_f\): time of first \texttt{FINISH} after start,
  \item runtime \(d = t_f - t_s\).
\end{itemize}

We filter tasks to those with \(d>0\), \(d \le 86400\) seconds, and \(0 < cpu\_req, mem\_req \le 1\).

\subsection{Trace subset and window selection}
To keep processing manageable, we download a subset of consecutive \texttt{task\_events} shards,
extract clean tasks, and select the busiest 2-hour window by maximizing arrivals in a sliding
2-hour window (1-minute bins). The selected busiest window contains 112,434 tasks.

\subsection{Summary statistics of the selected window}
For the busiest 2-hour window:
\begin{itemize}[leftmargin=1.2cm]
  \item number of tasks: 112,434,
  \item mean runtime: \(\approx 734.6\) seconds,
  \item mean CPU request: \(\approx 0.0477\),
  \item mean memory request: \(\approx 0.0329\).
\end{itemize}

% ============================================================
\chapter{System and Simulation Model}
\section{Resource model}
We simulate a pool of identical VMs, each with normalized capacity:
\[
C^{cpu}=1,\quad C^{mem}=1.
\]
A task \(i\) requests \((c_i, m_i)\) and occupies those resources for its runtime \(d_i\).

\section{Scheduling model}
Tasks arrive at \(t_a\) and join a FIFO queue if they cannot be placed immediately.
A task is placed on the first VM that can accommodate its CPU and memory requests (first-fit).
When a task finishes, it releases resources.

\section{Simulation approach}
We implement a discrete-event simulation:
\begin{itemize}[leftmargin=1.2cm]
  \item arrival events add tasks to the queue,
  \item completion events release resources and trigger additional placements,
  \item autoscaling decisions occur every \(\Delta=60\) seconds.
\end{itemize}

\section{Cost and performance metrics}
\subsection{Cost}
We measure cost as the time integral of active VM count:
\[
\text{VM-seconds} = \int_0^T k(t)\,dt,\qquad
\text{VM-hours} = \frac{\text{VM-seconds}}{3600}.
\]
Note: the simulation runs past the 2-hour arrival window until all tasks complete, so VM-hours
can exceed \(2 \times k\).

\subsection{Performance and SLA metrics}
We report:
\begin{itemize}[leftmargin=1.2cm]
  \item mean waiting time,
  \item p95 and p99 waiting time,
  \item SLA60 violation rate: \(P(W>60s)\),
  \item SLA120 violation rate: \(P(W>120s)\).
\end{itemize}

% ============================================================
\chapter{Autoscaling Policies}
\section{Static provisioning}
A fixed number of VMs \(k\) remain active for the entire simulation.

\section{Threshold autoscaling}
Every \(\Delta=60\) seconds we compute an aggregated queue backlog signal:
\[
Q(t) = \sum_{i \in \text{queue}} \max(c_i, m_i)\cdot d_i,
\]
interpretable as queued \emph{resource-seconds} of dominant demand.

If \(Q(t)\) exceeds an upper threshold, we scale up by a step size; if it falls below a lower
threshold, we scale down by a step size. Step sizes are used to ensure responsiveness under
bursty demand.

\section{MDP-based autoscaling (aggregated model)}
A full MDP state including per-VM allocations and remaining runtimes is too large for a short
project. Instead, we use an aggregated MDP approximation:
\begin{itemize}[leftmargin=1.2cm]
  \item State: \(\state=(k, \text{bin}(Q))\), where \(k\) is active VMs and \(\text{bin}(Q)\)
  discretizes queued work.
  \item Actions: \(\action\in\{-100, -50, 0, +50, +100\}\) VMs, clipped to \([k_{\min}, k_{\max}]\).
  \item Dynamics: \(Q_{t+1}=\max(0, Q_t + W^{in}_t - k_t\Delta)\), where \(W^{in}_t\) is sampled
  from trace-derived arrivals aggregated per control interval.
  \item Reward: negative weighted sum of VM cost, backlog, and action magnitude.
\end{itemize}
We learn the policy using tabular Q-learning on the aggregated MDP, then evaluate it in the
detailed event-driven simulator.

% ============================================================
\chapter{Experiments}
\section{Experimental setup}
We evaluate on the busiest 2-hour window extracted from the Google trace. Control interval is
\(\Delta=60\) seconds. The MDP training estimates capacity bounds \(k_{\min}=186\) and
\(k_{\max}=1200\) from trace-derived arrival work statistics.

\section{Policies evaluated}
\begin{itemize}[leftmargin=1.2cm]
  \item Static provisioning (baseline),
  \item Threshold autoscaling (heuristic),
  \item MDP autoscaling (learned policy).
\end{itemize}

% ============================================================
\chapter{Results and Discussion}

\section{Experimental context and metrics}
We evaluate three allocation strategies on the busiest 2-hour window extracted from the Google Cluster Trace
(112,434 tasks). The arrival window is 2 hours, but the simulation continues until all tasks complete; therefore,
cost is measured as VM-hours over the full completion horizon. Performance is evaluated via waiting-time percentiles
(p95/p99) and SLA violation rates, where SLA60 denotes \(P(W > 60\text{s})\) and SLA120 denotes \(P(W > 120\text{s})\).

\section{Overall cost--SLA trade-off}
Figure~\ref{fig:cost_vs_sla} summarizes the cost--SLA60 trade-off. The static provisioning sweep forms a
cost--SLA curve: increasing capacity (higher VM-hours) reduces the probability that tasks wait longer than 60 seconds.
Autoscaling policies shift the operating point by adapting capacity in response to bursts.

\begin{figure}[H]
  \centering
  \includegraphics[width=0.9\textwidth]{figures/cost_vs_sla.png}
  \caption{Cost vs SLA60 on the busiest 2-hour Google trace window. The static sweep forms a cost--SLA curve;
  threshold and MDP appear as operating points.}
  \label{fig:cost_vs_sla}
\end{figure}

\section{Main quantitative results}
Table~\ref{tab:main_results} reports cost and delay metrics for the three primary policies.

\begin{table}[H]
  \centering
  \caption{Trace-driven simulation results (Google trace, busiest 2-hour window; 112,434 tasks).}
  \label{tab:main_results}
  \begin{tabular}{lrrrrrr}
    \toprule
    Policy & VM-hours & Mean wait (s) & p95 (s) & p99 (s) & SLA60 & SLA120 \\
    \midrule
    Static (k=693) & 2407.0 & 47.23 & 442.50 & 685.41 & 0.1162 & 0.1146 \\
    Threshold & 2968.7 & 7.81 & 51.86 & 105.39 & 0.0428 & 0.0000 \\
    MDP & 2999.7 & 2.33 & 10.95 & 82.04 & 0.0109 & 0.0042 \\
    \bottomrule
  \end{tabular}
\end{table}

Overall, the fixed-capacity baseline (k=693) is cheapest among these three configurations but suffers substantially worse
tail delay and SLA violations during burst periods. Threshold autoscaling reduces both p95/p99 delays and SLA60 compared to
static, at a higher VM-hour cost. The MDP policy achieves the lowest waiting times and lowest SLA60 violation rate, at a cost
comparable to the threshold heuristic in this configuration.

\section{Matched-SLA comparisons (fair static baselines)}
A single static baseline can be misleading because static provisioning represents a family of policies parameterized by \(k\).
To ensure fair comparisons, we perform matched-SLA baselining: for each autoscaling policy we select a static capacity that
achieves approximately the same SLA60 violation rate, then compare cost and tail latency.

\begin{table}[H]
  \centering
  \caption{Matched-SLA comparisons against static provisioning (Google trace, busiest 2-hour window).}
  \label{tab:matched_sla}
  \begin{tabular}{lrrrr}
    \toprule
    Policy / Baseline & VM-hours & SLA60 & p99 wait (s) & Notes \\
    \midrule
    Threshold autoscaling & 2968.7 & 0.0428 & 105.39 & Heuristic control point \\
    Static (k=840) & 2917.6 & 0.0429 & 180.55 & Matches threshold SLA60 \\
    \midrule
    MDP autoscaling & 2999.7 & 0.0109 & 82.04 & Learned control point \\
    Static (k=890) & 3091.3 & 0.0108 & 66.02 & Matches MDP SLA60 \\
    \bottomrule
  \end{tabular}

  \vspace{0.2cm}
  \footnotesize
  \textbf{Note:} For high static capacities, p95 waiting time can become 0 because at least 95\% of tasks start immediately.
  Therefore, SLA violation rates and p99 waiting time provide a more informative fairness comparison.
\end{table}

At approximately the same SLA60 as the threshold autoscaler, Static(k=840) is slightly cheaper in VM-hours but exhibits
substantially worse tail latency (p99). At approximately the same SLA60 as the MDP policy, Static(k=890) requires more VM-hours,
while offering slightly lower p99 in this configuration. This indicates that adaptive policies can reduce cost at a given SLA target,
and that tail latency depends on how strongly the policy objective penalizes extreme backlog.

\section{Policy behavior over time}
Figures~\ref{fig:static_ts}--\ref{fig:mdp_ts} visualize the mechanism behind the metrics. Static provisioning keeps capacity constant,
so bursts create larger backlog that increases tail delays. Threshold and MDP autoscaling increase capacity during burst periods, reducing
queueing and improving SLA metrics.

\begin{figure}[H]
  \centering
  \includegraphics[width=0.9\textwidth]{figures/static.png}
  \caption{Static provisioning: constant VM capacity. Bursty arrivals create larger queued backlog and higher tail delays.}
  \label{fig:static_ts}
\end{figure}

\begin{figure}[H]
  \centering
  \includegraphics[width=0.9\textwidth]{figures/threshold.png}
  \caption{Threshold autoscaling: VMs increase in response to queued work and reduce backlog during bursts.}
  \label{fig:threshold_ts}
\end{figure}

\begin{figure}[H]
  \centering
  \includegraphics[width=0.9\textwidth]{figures/mdp.png}
  \caption{MDP autoscaling: learned actions respond more aggressively to backlog, reducing queueing and SLA violations.}
  \label{fig:mdp_ts}
\end{figure}

% ============================================================
\chapter{Limitations}
\begin{itemize}[leftmargin=1.2cm]
  \item \textbf{Homogeneous capacity:} real clusters have heterogeneous machines and VM types.
  \item \textbf{Requests vs usage:} we use requested CPU/memory, not actual utilization.
  \item \textbf{Ignored failure modes:} retries, evictions, and preemptions are filtered out.
  \item \textbf{Simplified scheduler:} FIFO + first-fit does not reproduce production scheduling constraints.
  \item \textbf{Aggregated MDP:} the MDP state compresses dynamics into a single queued-work signal.
\end{itemize}

% ============================================================
% ============================================================
\chapter{Dataset: Alibaba OpenB Pod Trace (Kaggle)}
\section{Source}
To satisfy the assignment requirement of using Alibaba traces, we use the \emph{Alibaba OpenB pod trace}
available via Kaggle \citep{alibabaopenbkaggle}. The dataset contains pod-level resource requests and lifecycle
timestamps, enabling conversion to the same task format used in our simulator.

\section{Raw schema used}
We use the following fields from the pod list tables (e.g., \texttt{openb\_pod\_list\_default.csv}):
\begin{itemize}[leftmargin=1.2cm]
  \item \texttt{cpu\_milli} (CPU request in millicores),
  \item \texttt{memory\_mib} (memory request in MiB),
  \item \texttt{creation\_time}, \texttt{scheduled\_time}, \texttt{deletion\_time} (timestamps),
\end{itemize}
and we use node capacity fields from \texttt{openb\_node\_list\_all\_node.csv}:
\texttt{cpu\_milli} and \texttt{memory\_mib}.

\section{Preprocessing and normalization}
We convert each pod record into:
\[
(\text{arrival\_time},\ \text{runtime},\ cpu\_req,\ mem\_req).
\]
We set:
\begin{itemize}[leftmargin=1.2cm]
  \item arrival time \(t_a\) = \texttt{creation\_time},
  \item start time \(t_s\) = \texttt{scheduled\_time} (fallback to \texttt{creation\_time} if missing/0),
  \item finish time \(t_f\) = \texttt{deletion\_time},
  \item runtime \(d = t_f - t_s\).
\end{itemize}

To match our homogeneous VM simulator, we normalize CPU and memory requests by node capacities observed in the trace:
\(C^{cpu} = 32000\) millicores and \(C^{mem} = 262144\) MiB (from \texttt{openb\_node\_list\_all\_node.csv}),
so that \(cpu\_req, mem\_req \in (0,1]\). We filter to pods with positive runtime and requests that fit within
a single normalized VM.

\section{Window selection}
Unlike the Google trace, this Alibaba OpenB dataset is relatively sparse over time. A 2-hour window contained
too few events for stable autoscaling evaluation, so we select the busiest 24-hour window using a sliding-window
arrival count. The selected busiest 24-hour window contains 681 pods.


% ============================================================
\chapter{Results on Alibaba Trace (OpenB Pods)}
\section{Main quantitative results}
Table~\ref{tab:alibaba_main_results} summarizes cost and performance on the busiest 24-hour Alibaba window (681 pods).
Because many pods start immediately when sufficient capacity is provisioned, p95 can become 0; therefore we also report
SLA violation rates and p99.

\begin{table}[H]
  \centering
  \caption{Trace-driven simulation results (Alibaba OpenB pods, busiest 24-hour window; 681 pods).}
  \label{tab:alibaba_main_results}
  \begin{tabular}{lrrrrrr}
    \toprule
    Policy & VM-hours & Mean wait (s) & p95 (s) & p99 (s) & SLA60 & SLA120 \\
    \midrule
    Static (k=83) & 2596.7 & 0.00 & 0.00 & 0.00 & 0.0000 & 0.0000 \\
    Threshold & 144.6 & 35.57 & 198.00 & 889.80 & 0.1175 & 0.0808 \\
    MDP & 202.4 & 1.46 & 0.00 & 80.40 & 0.0147 & 0.0000 \\
    \bottomrule
  \end{tabular}
\end{table}

The threshold autoscaler achieves a large cost reduction compared to static provisioning, but incurs higher tail latency
and SLA violations. The MDP policy reduces SLA60 substantially (from 11.75\% to 1.47\%) and lowers p99 waiting time,
at a moderate increase in VM-hours relative to the threshold baseline.

\section{Cost--SLA trade-off}
Figure~\ref{fig:alibaba_cost_vs_sla} plots VM-hours versus SLA60 for a static capacity sweep and overlays the threshold
and MDP operating points.

\begin{figure}[H]
  \centering
  \includegraphics[width=0.9\textwidth]{figures/alibaba_cost_vs_sla.png}
  \caption{Cost vs SLA60 on the Alibaba OpenB pod trace (busiest 24-hour window).}
  \label{fig:alibaba_cost_vs_sla}
\end{figure}

\section{Temporal behavior}
Figures~\ref{fig:alibaba_static_ts}--\ref{fig:alibaba_mdp_ts} illustrate how the policies respond over time.

\begin{figure}[H]
  \centering
  \includegraphics[width=0.9\textwidth]{figures/alibaba_static.png}
  \caption{Alibaba: Static provisioning (fixed VM capacity).}
  \label{fig:alibaba_static_ts}
\end{figure}

\begin{figure}[H]
  \centering
  \includegraphics[width=0.9\textwidth]{figures/alibaba_threshold.png}
  \caption{Alibaba: Threshold autoscaling response over time.}
  \label{fig:alibaba_threshold_ts}
\end{figure}

\begin{figure}[H]
  \centering
  \includegraphics[width=0.9\textwidth]{figures/alibaba_mdp.png}
  \caption{Alibaba: MDP autoscaling response over time.}
  \label{fig:alibaba_mdp_ts}
\end{figure}


% ============================================================
\chapter{Conclusion}
This project demonstrates how stochastic modeling and MDP-based decision-making can guide cloud resource
allocation under uncertainty. Using a trace-driven workload from the Google Cluster Trace and an event-driven
simulator, we quantify the cost--performance trade-offs between static provisioning, heuristic threshold scaling,
and an MDP autoscaler. Results show that autoscaling policies can dramatically reduce waiting times and SLA
violations during bursts, and an MDP policy can further reduce SLA60 at comparable cost by adapting capacity
to the workload over time.

\bibliographystyle{plainnat}
\bibliography{refs}

\appendix
\chapter{Reproducibility Notes}
\begin{itemize}[leftmargin=1.2cm]
  \item Data extraction: DuckDB SQL to build a clean task table from raw \texttt{task\_events}.
  \item Simulator: Python discrete-event engine with CPU+memory constraints.
  \item Figures: generated by scripts in \texttt{src/} and saved to \texttt{figures/}.
\end{itemize}

\end{document}